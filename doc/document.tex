
\documentclass[en, bibend=bibtex]{elegantpaper}

\title{KoopmanDL Documentation}
\author{Yixiao Qian}
\institute{Zhejiang University}
\date{\today}
\setcounter{tocdepth}{3}
% \everymath{\displaystyle}
\usepackage{esint}
\theoremstyle{plain}
\usepackage{miniplot}
\usepackage{algorithm}
\usepackage{algorithmic}

\renewcommand{\proofname}{\textsl {Proof}}

\usepackage{bm}
\makeatletter
\def\renewtheorem#1{%
  \expandafter\let\csname#1\endcsname\relax
  \expandafter\let\csname c@#1\endcsname\relax
  \gdef\renewtheorem@envname{#1}
  \renewtheorem@secpar
}
\def\renewtheorem@secpar{\@ifnextchar[{\renewtheorem@numberedlike}{\renewtheorem@nonumberedlike}}
\def\renewtheorem@numberedlike[#1]#2{\newtheorem{\renewtheorem@envname}[#1]{#2}}
\def\renewtheorem@nonumberedlike#1{
  \def\renewtheorem@caption{#1}
  \edef\renewtheorem@nowithin{\noexpand\newtheorem{\renewtheorem@envname}{\renewtheorem@caption}}
  \renewtheorem@thirdpar
}
\def\renewtheorem@thirdpar{\@ifnextchar[{\renewtheorem@within}{\renewtheorem@nowithin}}
\def\renewtheorem@within[#1]{\renewtheorem@nowithin[#1]}
\makeatother

\begin{document}
\maketitle

\section{Dictionary}

\subsection{RBFDictionary}

\begin{definition}
  Let $(V, \|\cdot\|)$ be a normed linear space,
  a function $\varphi: V \rightarrow \mathbb{R}$ is called
  a \emph{radial basis function (RBF)} if
  there exists a univariate function $\hat{\varphi}: [0, +\infty) \rightarrow \mathbb{R}$
  such that for all $\mathbf{x} \in V$
  \begin{equation*}
   \varphi(\mathbf{x}) = \hat{\varphi}(\|\mathbf{x} - \mathbf{c}\|),
  \end{equation*}
  for some center point $\mathbf{c} \in V$.
\end{definition}

Assume we want to build a RBF Dictionary
with $M$ RBF centers and a regularizer $\lambda \in \mathbb{R}$.
Given a dataset $X \in \mathbb{R}^{N \times D}$,
where $N$ is the number of samples and
$D$ is the number of features,
we first apply k-means clustering to find $M$ centers:
\begin{equation*}
  \{\mu_1, \mu_2, \cdots, \mu_M\} \subset \mathbb{R}^D.
\end{equation*}
For each data point $\mathbf{x}_i \in X$,
the distance between $\mathbf{x}_i$ and $\mu_j$ is
$r_{ij} = \|\mathbf{x}_i - \mu_j\|$,
then the RBF value is computed as
\begin{equation*}
  \phi_{ij} = (r_{ij})^2 \log(r_{ij} + \lambda),
\end{equation*}
where $\phi_{ij}$ is the RBF value for the
$i$-th data point and the $j$-th center.
Collecting all the RBF values yields a matrix $\Phi \in \mathbb{R}^{N \times M}$
\begin{equation*}
  \Phi = \left[
    \begin{array}{cccc}
      \phi_{11}&\phi_{12}&\cdots&\phi_{1M}\\
      \phi_{21}&\phi_{22}&\cdots&\phi_{2M}\\
      \vdots & \vdots & \ddots & \vdots\\
      \phi_{N1}&\phi_{N2}&\cdots&\phi_{NM}
    \end{array}
  \right].
\end{equation*}
To keep the same as Trainable Dictionary,
we add one column of all-one vector and
the projection part
\begin{equation*}
  \text{output} = \left[
    \begin{array}{ccc}
      \mathbf{1}&X&\Phi
    \end{array}
  \right],
\end{equation*}
where $\mathbf{1} \in \mathbb{R}^{N\times 1}$ is a vector with all elements
equal to one.




\end{document}

%%% Local Variables:
%%% mode: latex
%%% TeX-master: t
%%% End:


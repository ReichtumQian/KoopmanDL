

\section{Dictionary}
\subsection{Class Dictionary}

\begin{itemize}
\item Brief Description: A \lstinline|Dictionary| is a batched vector function,
  represented by $\Psi: \mathbb{R}^{N \times d} \rightarrow
  \mathbb{R}^{N \times M}$,
  where $N$ is the size of the dataset.
\item Attributes:
  \begin{itemize}
  \item \lstinline|_func|: The batched vector function.
  $\Psi: \mathbb{R}^{N \times d} \rightarrow \mathbb{R}^{N \times M}$.
  \item \lstinline|_M|: The output dimension of the vector function.
  \end{itemize}
\item Methods:
  \begin{itemize}
  \item \lstinline|__init__(self, M, func)|: Initializes the dictionary
  with output dimension \lstinline|M| and vector function \lstinline|func|.
  \item \lstinline|__call__(self, x)|: Applies the dictionary to input \lstinline|x|,
    which must satisfy $x \in \mathbb{R}^{N \times d}$.
  \end{itemize}
\end{itemize}

\subsection{Class TrainableDictionary(Dictionary)}

\begin{itemize}
\item Brief Description: A \lstinline|TrainableDictionary|
  is a dictionary containing a trainable neural network,
  mapping $\mathbb{R}^{N \times d} \rightarrow \mathbb{R}^{N \times M}$.
  The output includes non-trainable components $\mathbf{1}$ and $x$.
\item Attributes:
  \begin{itemize}
  \item \lstinline|__optimizer|: The optimizer of the trainable neural network.
  \end{itemize}
\item Methods:
  \begin{itemize}
  \item \lstinline|__init__(self, M, func, optimizer)|:
  Initializes the trainable dictionary.
  The \lstinline|func| must be an instance of \lstinline|torch.nn.Module|.
  \item \lstinline|train(self, data_loader, loss_func, n_epochs)|:
  Trains the neural network using the provided data loader, loss function,
  and number of epochs.
  \end{itemize}
\end{itemize}

\subsection{Class RBFDictionary(Dictionary)}

\begin{itemize}
\item Brief Description:
  An \lstinline|RBFDictionary| is a dictionary
  whose basis functions are radial basis functions (RBFs).
  The output includes non-trainable components $\mathbf{1}$ and $x$.
\item Attributes:
  \begin{itemize}
  \item \lstinline|__regularizer|: The regularization factor for the RBFs.
  \end{itemize}
\item Methods:
  \begin{itemize}
  \item \lstinline|build(self, data)|: Builds the basis functions and the dictionary
  using the provided data.
  \end{itemize}
\end{itemize}

\section{Solver}

\subsection{Class EDMDSolver}

\begin{itemize}
\item Brief Description: This class implements the EDMD algorithm.
  It also serves as the base class for \lstinline|EDMDDLSolver|
\item Attributes:
  \begin{itemize}
  \item \lstinline|_dictionary|:
    A dictionary composed of basis functions.
    It can be an instance of \lstinline|Dictionary|
    or \lstinline|RBFDictionary|,
    but not \lstinline|TrainableDictionary|.
  \item \lstinline|eigenvalues|: Stores the eigenvalues of $K$.
  \item \lstinline|right_eigenvectors|: A 2D tensor where each column represents a right eigenvector of the matrix $K$.
  \item \lstinline|left_eigenvectors|: A 2D tensor where each column represents a left eigenvector of the matrix $K$.
  \end{itemize}
\item Methods:
  \begin{itemize}
  \item \lstinline|__init__(self, dictionary)|:
    Initializes the \lstinline|EDMDSolver| with the specified dictionary.
  \item \lstinline|compute_K(self, data_x, data_y)|:
    Computes the matrix $K$ using the formula $K = (AG^+)^T$,
    where $G^+$ denotes the pseudoinverse of $G$.
  \item \lstinline|compute_eig(self, K)|: Computes the eigenvalues,
    left eigenvectors and right eigenvectors of the matrix $K$.
  \item \lstinline|predict(self, x0, traj_len)|: 
    Predicts the trajectory of the system starting from $x_0$
    with length $traj\_len$.
  \item \lstinline|save(self, path)|: Saves the solver to the specified path.
  \item \lstinline|load(self, path)|: Loads the solver from the specified path.
  \end{itemize}
\end{itemize}

\subsection{Class EDMDDLSolver(EDMDSolver)}

\begin{itemize}
\item Brief Description: This class implements the EDMD-DL algorithm.
\item Attributes:
  \begin{itemize}
  \item \lstinline|__regularizer|: the regularizer $\lambda$
    used in the computation of $K$.
  \end{itemize}
\item Methods:
  \begin{itemize}
  \item \lstinline|__init__(self, dictionary, regularizer)|:
    Initializes the \lstinline|EDMDDLSolver| with the specified dictionary
    and regularization parameter.
  \item \lstinline|compute_K(self, data_x, data_y)|:
    Computes the matrix $K$ using the formula $K^T = A(G + \lambda I)^+$,
    where $I$ is the identity matrix and $\lambda$ is the regularization parameter.
  \item \lstinline|solve(self, data_x, data_y, n_epochs, batch_size, tolerance)|:
    Applies the EDMD-DL algorithm to solve the system,
    iterating for a specified number of epochs, batch size and tolerance.
  \end{itemize}
\end{itemize}

\section{Neural Networks}

\subsection{Class TanhResBlock(torch.nn.Module)}

\begin{itemize}
\item Brief Description: A residual block with tanh activation function.
\item Attributes:
  \begin{itemize}
  \item \lstinline|__linear1|: A fully connected (linear) layer that maps the input to an intermediate representation.
  \item \lstinline|__tanh|: The tanh activation function applied to the output of the first linear layer.
  \item \lstinline|__linear2|: A second fully connected layer that maps the activated intermediate representation back to the original input dimension.
  \item \lstinline|__shortcut|: A mechanism to bypass the block by adding the input directly to the output of the second linear layer, facilitating the residual connection.
  \end{itemize}
\item Methods:
  \begin{itemize}
  \item \lstinline|forward(x)|:
  Computes the forward pass of the block. 
  Applies the first linear transformation, followed by the tanh activation, 
  a second linear transformation, 
  and finally adds the input to the output (residual connection).
  \end{itemize}
\end{itemize}

\subsection{Class TanhResNet(torch.nn.Module)}

\begin{itemize}
\item Brief Description: A simple full-connected network with tanh activation,
  achieved by PyTorch.
\item Attributes:
  \begin{itemize}
  \item \lstinline|__network|: 
  A sequential container of \lstinline|TanhResBlock| instances 
  forming the network's architecture.
  \end{itemize}
\item Methods:
  \begin{itemize}
  \item \lstinline|__init__(self, input_dim, output_dim, hidden_layer_sizes)|: Initializes the network. Constructs a series of \lstinline|TanhResBlock| layers based on the specified input dimension, output dimension, and sizes of hidden layers.
  \item \lstinline|forward(self, x)|: Executes the forward pass through the network, passing the input through each layer in the sequence.
  \end{itemize}
\end{itemize}

\subsection{Class TanhNetWithNonTrainable(TanhResNet)}

\begin{itemize}
\item Brief Description: 
  An extension of the \lstinline|TanhResNet| class designed to 
  integrate with a \lstinline|TrainableDictionary|. 
  This network includes both trainable and non-trainable components.
\item Methods:
  \begin{itemize}
  \item \lstinline|__init__(self, input_dim, output_dim, hidden_layer_sizes, n_nontrainable)|
    Initializes the network with specified dimensions. 
    The parameter \lstinline|n_nontrainable| determines the number of outputs 
    from the non-trainable layer.
  \item \lstinline|forward(self, x)|: Computes the forward pass through the network, 
    including the non-trainable layer.
  \end{itemize}
\end{itemize}

\section{ODE and Flowmap}

\subsection{Class ODE}

\begin{itemize}
\item Brief Description: An abstract basis class for ordinary
  differential equations (ODEs).
  \begin{equation*}
    \dot{x}(t) = f(x(t)).
  \end{equation*}
\item Attributes:
  \begin{itemize}
  \item \lstinline|_dim|: The dimension of the ODE system.
  \item \lstinline|_rhs|: A function representing the right-hand side of the ODE.
  \end{itemize}
\item Methods:
  \begin{itemize}
  \item \lstinline|__init__(self, rhs, dim)|: Initializes the ODE 
    with the specified right-hand side function and dimension.
  \end{itemize}
\end{itemize}

\subsection{Class DuffingOscillator}

\begin{itemize}
\item Brief Description: A concrete implementation of the Duffing Oscillator model.
  \begin{align*}
    &\dot{x}_1 = x_2,\\
    &\dot{x}_2 = -\delta x_2 - x_1(\beta + \alpha x_1^2).
  \end{align*}
\item Methods:
  \begin{itemize}
  \item \lstinline|__init__(self, alpha, beta, delta)|: Initializes the Duffing Oscillator
    with parameters \lstinline|alpha|,
    \lstinline|beta|, and \lstinline|delta|.
  \end{itemize}
\end{itemize}

\subsection{Class VanDerPolOscillator}

\begin{itemize}
\item Brief Description: A concrete implementation of the Van der Pol Oscillator model.
  \begin{align*}
    &\dot{x}_1 = x_2,\\
    &\dot{x}_2 = \alpha(1 - x_1^2)x_2 - x_1.
  \end{align*}
\item Methods:
  \begin{itemize}
  \item \lstinline|__init__(self, alpha)|: Initializes the Van der Pol Oscillator with parameter \lstinline|alpha|.
  \end{itemize}
\end{itemize}

\subsection{Class FlowMap}

\begin{itemize}
\item Brief Description: Implements the flow map $\varphi_t(x) := x(t), x(0) = x$. 
  This class serves as a base for numerical solvers for initial value problems (IVP).
\item Attributes:
  \begin{itemize}
  \item \lstinline|_dt|: The time step size used in the numerical solver.
  \end{itemize}
\item Methods:
  \begin{itemize}
  \item \lstinline|step(self, ode, x_init)|: Advances the solution of the ODE by one time step from the initial state \lstinline|x_init|.
  \end{itemize}
\end{itemize}

\subsection{Class ForwardEuler(FlowMap)}

\begin{itemize}
\item Brief Description: A concrete implementation of the forward Euler method
 for solving IVPs.
 The update formula is $x(n+1) = x(n) + kf(x(n))$,
 where $k$ is the time step.
\item Methods:
  \begin{itemize}
  \item \lstinline|step(self, ode, x_init)|:
    Overrides the base class method to implement the forward Euler update.
  \end{itemize}
\end{itemize}



